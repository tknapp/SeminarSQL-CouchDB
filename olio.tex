\section{Eine typische Web 2.0-Anwendung: Olio}
\label{olio}

Der Wechsel von einer relationalen Datenbank zu einer nicht-relationalen Datenbank soll nicht abstrakt, sondern konkret anhand einer Beispiel-Web-Anwendung evaluiert werden. Die hier verwendete Beispielanwendung nennt sich \textit{Olio}. Zur Evaluation wurde diese Anwendung ausgew�hlt, weil sie typische Eigenschaften einer Web 2.0-Anwendung wie Nutzerinteraktion und dynamische Inhalte aufweist und au�erdem frei zug�nglichen Quellcode besitzt. Die Anwendung wurde im Rahmen des Projektes Cloudstone urspr�nglich als Referenzanwendung f�r Performanztests entwickelt. Olio liegt in verschiedenen Implementierungen vor (Ruby, PHP und JavaEE) vor und kann auf verschiedenen Hardware-Konfigurationen bereitgestellt werden. \\
Inhaltlich handelt es sich bei Olio um eine Anwendung zur Ver�ffentlichung von sozialen Ereignissen aller Art. Besucher der Anwendungen k�nnen Ereignisse anschauen und kommentieren, eigene Nutzerkonten anlegen und selbst Ereignisse ver�ffentlichen. Au�erdem kann ein Nutzer an einem Ereignis "`teilnehmen"', indem er sein Nutzerkonto mit dem Ereignis verbindet. Andere Nutzer k�nnen so sehen, wer voraussichtlich eine Veranstaltung besuchen wird. Sind viele Nutzer mit einem Ereignis verbunden, wird es auch f�r andere Nutzer attraktiver. Es entstehen die f�r soziale Netzwerkeffekte so typischen Netzwerkeffekte.\\

F�r die Evaluation bez�glich der Austauschbarkeit der Datenbanken wurde die Ruby-Implementierung von Olio gew�hlt. Olio wurde in diesem Fall als Ruby On Rails-Anwendung entwickelt. \textit{Rails} \cite{rubyOnRailsWebsite} ist ein in Ruby geschriebenes Open-Source Rahmenwerk zur Entwicklung agiler Web-Anwendungen, welches die Direktheit von PHP und die Architektureigenschaften von Java vereint \cite{RoRArticle}. Es stellt Konvention �ber Konfiguration und baut auf dem Architekturstil \textit{Model View Controller (MVC)} auf. Dieser Stil besteht im Wesentlichen aus drei Komponenten, die in Tabelle \ref{tab:KomponentenDesArchitekturstilsTextitModelViewController} vorgestellt werden.

\begin{table}[hp]
	\centering
		\begin{tabular}{|l|p{9cm}|}
		\hline
		\textbf{Models} & Die Daten der Web-Anwendung sowie die Regeln zur Manipulation dieser Daten werden hier verwaltet. Diese Komponente stellt die Verbindung zur verwendeten Datenbank her.\\
		\hline
		\textbf{Views} & Sie stellen die Nutzerschnittstelle dar und k�mmern um die Versorgung der Klienten mit Daten. Ein Klient k�nnte hierbei ein Webbrowser oder eine andere Anwendung sein. Daten sind meist vorverarbeitete Daten, beispielsweise in Form einer generierten HTML-Seite. \\
		\hline
		\textbf{Controller  } & Diese Komponente koordiniert das Zusammenspiel von Models und Views und stellt somit den "`Klebstoff"' zwischen den beiden Komponenten dar. Es nimmt Anfragen von au�en an, leitet diese entsprechend weiter und regelt das Versenden der entsprechenden Antworten.\\
		\hline
		\end{tabular}
	\caption{Komponenten des Architekturstils \textit{Model View Controller}}
	\label{tab:KomponentenDesArchitekturstilsTextitModelViewController}
\end{table}

Ruby on Rails-Anwendungen sind grunds�tzlich f�r die Nutzung von relationalen Datenbanken vorkonfiguriert. Hierzu wurde das Konzept \textit{ActiveRecord} entworfen, welches eine Schnittstelle zu einer relationale Datenbank bereitstellt. Da Ruby eine objektorientierte Programmiersprache ist, k�nnen die Ergebnisse von SQL-Abfragen nicht direkt verarbeitet werden, da diese nur als Zeichenkette interpretiert w�rden. Zur �berbr�ckung dieser Problematik bildet ActiveRecord SQL-Tabellen auf Ruby-Klassen ab und einzelne Zeilen auf Ruby-Objekte. So k�nnen die in Tabellen dargestellten Ergebnisse von SQL-Abfragen in Ruby bearbeitet werden.\\
Da ActiveRecord allerdings nur f�r Datenbanken mit relationalem Tabellenschemata verwendet werden kann, wird der Anschluss von CouchDB mit seinem schemalosen Design auf diese Weise nicht funktionieren. Der Umbau von einer relationalen auf eine nicht-relationale Datenbank verursacht folglich einen h�herem Aufwand als die blo�e Anpassung einer Konfigurationsdatei zum Anschluss der Datenbank.