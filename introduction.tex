
\section{Einleitung}

Seit der Kommerzialisierung des Internets in den fr�hen 90er Jahren haben sich sowohl die Inhalte als auch deren Darstellung grundlegend ver�ndert. Anfangs bestand es lediglich aus statischen, textlastigen Seiten, die kaum Bilder enthielten und mit einfacher Struktur den geringen verf�gbaren Bandbreiten gerecht wurden. Dieses sogenannte Web 1.0 wandelte sich im Laufe der Zeit zum heutigen Web 2.0, in dem der Internetnutzer nicht mehr nur Informationen konsumiert, sondern aktiv an der Gestaltung der Inhalte beteiligt ist. Foren, Blogs und soziale Netzwerke sorgen heute mit aufwendigen, dynamischen Inhalten f�r betr�chtliche Datenmengen, f�r die immer h�here Bandbreiten ben�tigt werden. Dynamische Inhalte in Webseiten zu verwenden bedeutet vor allem, dass h�ufig Anfragen an Datenbanken gemacht werden m�ssen, in denen die einzelnen Elemente einer Seite wie z.B. Bilder, Texte oder Dokumente abgelegt sind. Durch die heute verf�gbaren hohen Bandbreiten hat sich der Performanz-Flaschenhals vom Netzwerk hin zur Datenbank verlagert. Gerade bei aufw�ndigen Anfragen liegt die Zeit zur Verarbeitung der Anfrage in der Datenbank oft deutlich �ber dem, was das Netzwerk zulassen w�rde (\cite[S. 11]{PerformanceComparison}, \cite{ActiveQueryCaching}). \\
Meist wurden bei Web-Anwendungen bisher relationale Datenbanken wie MySQL, SQLite oder  PostgreSQL eingesetzt. Die M�chtigkeit dieser Datenbanken vor allem in Bezug auf die Vielfalt der Anfragem�glichkeiten ist unbestritten. Jedoch kann gerade diese Vielfalt bei Web-Anwendungen zu Problemen f�hren, wenn Anfragen sehr komplex sind und somit zu lange brauchen. Wird bei einer Datenbank zus�tzlich auf Konsistenz gesetzt (ein Zustand, indem alle Klienten einer Datenbank zu einem bestimmten Zeitpunkt die gleichen Daten sehen), kann dies zu weiteren Problemen bei der Verf�gbarkeit eines Dienstes f�hren. Denn Konsistenz und Verf�gbarkeit k�nnen nach dem CAP-Theorem nicht gleichzeitig erreicht werden \cite[S. 12]{couchDB}.\\
Sowohl in der Wirtschaft als auch in der Wissenschaft sind in Folge ver�nderter Anforderungen durch das Internet verschiedene Datenbankmodelle entwickelt worden, die nicht mehr einem streng relationalen Ansatz folgen. Amazon beispielsweise hat mit Dynamo \cite{dynamo} eine Datenbank entwickelt, die vor allem die Verf�gbarkeit der Inhalte auf Kosten der Konsistenz in den Vordergrund stellt. Auch Google hat mit BigTable \cite{bigtable} ein neues Datenbankkonzept zur Speicherung rie�iger verteilter Datenmengen geschaffen. Alternative Datenbank aus dem wissenschaftlichen Umfeld sind beispielsweise CouchDB \cite{couchDB} oder MongoDB \cite{mongoDB}. Die Umstellung einer Web-Anwendung, der bisher eine relationale Datenbank zu Grunde lag, gestaltet sich jedoch meist schwierig. \\

Diese Arbeit untersucht die M�glichkeit der Umstellung von einem relationalen Datenbanksystem hin zu einem alternativen Datenbanksystem am Beispiel einer typischen Web 2.0-Anwendung. Die hier betrachtete Web-Anwendung nennt sich \textit{Olio} und wurde im Rahmen des \textit{Project Cloudstone} \cite{Cloudstone} entwickelt. Hierbei handelt es sich um die Implementierung einer Webseite zur Ver�ffentlichung von sozialen Ereignissen, an denen Nutzer wie in einem sozialen Netzwerk teilnehmen k�nnen, Ereignisse kommentieren oder eigene Ereignisse ver�ffentlichen k�nnen. Implementiert wurde die Anwendung auf Basis einer MySQL-Datenbank, welche nun durch eine Couch-Datenbank ersetzt werden soll.\\

Der Leser wird zun�chst in Kapitel \ref{couchDB} am Beispiel von CouchDB in die Welt der nicht-relationalen Datenbanken eingef�hrt. Kapitel \ref{olio} beschreibt die zu betrachtende Beispielanwendung Olio, deren Architektur und die geplante Art der Bereitstellung. Nachdem das grunds�tzliche Verst�ndnis f�r die Komponenten geschaffen wurde, beschreibt Kapitel \ref{analyse} zun�chst die Anforderungen an eine Datenbankumstellung, nennt m�gliche Herausforderungen und stellt schlie�lich ein Vorgehen zur praktischen Umstellung der Gesamtanwendung vor. Kapitel \ref{datenmigration} besch�ftigt sich mit der �bertragung von bereits vorhandenen Daten aus einer MySQL-Datenbank in eine Couch-Datenbank und stellt ein Werkzeug vor, mit dem die �bertragung automatisiert werden kann. Das letzte Kapitel, \ref{zusammenfassung}, fasst die gewonnen Erkenntnisse zusammen und zieht ein Fazit.


\textcolor{red}{REMOVE}
\cite{couchRest}
